\title{Dependent variable definitions}
\begin{description}
\item[Life expectancy at birth]: Number of years a newborn infant could expect to live if prevailing patterns of age-specific mortality rates at the time of birth stay the same throughout the infant’s life.
\item[Mean years of schooling]: Average number of years of education received by people ages 25 and older, converted from educational attainment levels using official durations of each level.
\item[Expected years of schooling]: Number of years of schooling that a child of school entrance age can expect to receive if prevailing patterns of age-specific enrolment rates persist throughout life
\item[Gross national income (GNI) per capita]: Aggregate income of an economy generated by its production and its ownership of factors of production, less the incomes paid for the use of factors of production owned by the rest of the world, converted to international dollars using PPP rates, divided by midyear population.
\item[Quintile income ratio]:Ratio of the average income of the richest 20\% of the population to the average income of the poorest 20\% of the population.
\item[Income Gini coeffcient]: Measure of the deviation of the distribution of income among individuals or households within a country from a perfectly equal distribution. A value of 0 represents absolute equality, a value of 100 absolute inequality.
\item[Maternal mortality ratio]: Ratio of the number of maternal deaths to the number of live births in a given year, expressed per 100,000 live births.
\item[Adolescent fertility rate]: Adolescent fertility rate is the number of births per 1,000 women ages 15-19.
\item[Labour force participation rate]: Proportion of a country’s working-age population (ages 15 and older) that engages in the labour market, either by working or actively looking for work, expressed as a percentage of the working-age population.
\end{description}